随着人工智能技术的飞速发展,多模态大模型在复杂交互场景中的应用逐渐受到关注。KFC点餐场景作为一个典型的高频、多模态交互
场景,涉及语音、文本、图像等多种信息的处理,对大模型的后训练提出了新的挑战和机遇。本文聚焦于面向KFC点餐的多模态大模型
的后训练研究,旨在通过优化模型对多模态信息的理解和生成能力,提升点餐系统的智能化水平和用户体验。通过系统性地分析多模态
后训练策略,并结合实验验证,本文为大模型在快餐行业中的应用提供了理论支持和实践指导,同时也为多模态交互场景中的模型优化
提供了新的思路。
