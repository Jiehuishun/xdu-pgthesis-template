\documentclass{xdupgthesis}
\usepackage{multirow}
\xdusetup{
    style = { cjk-font = fandol, latin-font = gyre, biblatex-option = { gbnamefmt = quanpin }},
    info = {
        title = {面向KFC点餐的多模态大模型的后训练研究},
        title* = {Post-training research on multimodal large model for KFC ordering},
        author = {上校},
        author* = {Colonel},
        department = {杭州研究院},
        abstract = {chapters/abstract-zh.tex},
        abstract* = {chapters/abstract.tex},
        keywords = {多模态大模型,后训练,KFC点餐},
        keywords* = {Multimodal LLM,Post-training,KFC Ordering},
        graduate-type = {硕士},
        degree-type = {专业},
        degree = {电子信息硕士},
        degree* = {Electronic Information},
        domain = {计算机技术},
        supervisor = {麦克阿瑟},
        supervisor* = {MacArthur},
        supv-ent = {艾森豪威尔},
        supv-ent* = {Eisenhower},
        supv-title = {教授},
        supv-title* = {Associate Professor},
        supv-ent-title = {高工},
        supv-ent-title* = {Senior Engineer},
        student-id = {22030101001},
        clc = {TP301.6},
        secret-level = {公开},
        bib-resource = {references.bib,},
        }
}
\begin{document}

\chapter{引言}

\section{研究背景}

在快节奏的现代生活中,快餐行业的用户体验和运营效率至关重要。KFC作为全球知名的快餐品牌,其点餐系统不仅需要高效处理用户的订单需求,
还需要适应多样化的交互方式,如语音点餐、图像识别菜单等。随着多模态大模型技术的兴起,如何通过后训练方法优化模型对多模态信息的理解和
生成能力,成为提升点餐系统智能化水平的关键。

\section{研究意义}

本研究旨在通过多模态后训练方法,提升大模型在KFC点餐场景中的表现,优化用户体验并提高点餐效率。这不仅为快餐行业提供了新的技术解决
方案,也为多模态交互场景中的模型优化提供了理论支持。


\section{研究内容与结构安排}


\subsection{研究内容}

本文将系统研究多模态大模型的后训练方法,分析其在KFC点餐场景中的应用效果,并通过实验验证优化策略的有效性。


\subsection{结果安排}

全文共分为六个部分,详细阐述了研究背景、相关工作、研究方法、实验设计与结果分析、讨论与展望以及结论。

\chapter{相关工作}

\section{多模态大模型技术综述}

近年来,多模态大模型在自然语言处理、计算机视觉等领域取得了显著进展。多模态模型能够同时处理文本、图像、语音等多种信息,展现
出强大的交互能力和适应性。然而,如何针对特定场景优化多模态模型的性能,仍是当前研究的热点问题。

\section{KFC点餐系统的多模态需求}

KFC点餐场景涉及多种交互方式,如语音点餐、菜单图像识别、文本交互等。传统的点餐系统难以满足这些复杂需求,而多模态大模型的引入
为提升点餐系统的智能化水平提供了可能。

此外,人们也讨论了快餐食品与道德消费之间的关系,认为快餐食品的生产和消费行为应该符合道德标准,保护环境和动物权益\cite{schroder2005fast}。


\chapter{研究方法}

\section{多模态后训练框架设计}

本文设计了一种面向KFC点餐的多模态后训练框架,结合监督微调、偏好学习和强化学习等技术,优化模型对多模态信息的理解和生成能力。

\section{数据收集与预处理}

为了支持多模态后训练,本文收集了包括语音、图像和文本在内的多模态数据,并通过数据清洗、标注等步骤构建高质量的训练数据集。

\section{训练策略与优化方法}

本文采用迭代式的后训练策略,通过监督微调优化模型的文本生成能力,通过偏好学习提升模型对用户偏好的适应性,通过强化学习优化模型的交互策略。

\chapter{实验设计与结果分析}

\section{实验环境与数据集}

实验基于常见的多模态大模型架构进行,数据集包括用户点餐对话记录、菜单图像和语音数据。


\section{实验结果与分析}

通过对比不同后训练策略的效果,实验结果表明,结合多种后训练技术的模型在点餐场景中表现优异,显著提升了点餐效率和用户满意度。

\section{模型性能对比}

本文还对比了优化前后模型的性能,结果表明后训练方法显著提升了模型的多模态交互能力和用户体验。


\chapter{讨论与展望}
5. 讨论与展望

\section{研究贡献}

本研究提出了一种面向KFC点餐的多模态后训练框架,显著提升了模型在多模态交互场景中的表现,为快餐行业的智能化点餐系统提供了新的技术路径。

\section{研究局限性}

尽管取得了显著成果,但研究仍存在数据规模有限、模型泛化能力不足等问题,未来需要进一步优化。

\section{未来工作方向}

未来,我们将扩大数据规模,探索多模态信息的深度融合,并研究模型在跨场景中的泛化能力,进一步推动多模态大模型在快餐行业中的应用。

\chapter{结论}
6. 结论
本文针对KFC点餐场景,提出了一种面向多模态大模型的后训练方法,并通过实验验证了其有效性。研究结果表明,优化后的模型在多模态交互场景
中表现优异,显著提升了用户体验和点餐效率。未来,我们将通过扩大数据规模和优化后训练策略,进一步提升模型的泛化能力和智能化水平。



\end{document}